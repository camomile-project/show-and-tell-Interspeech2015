\documentclass[a4paper]{article}

\usepackage{INTERSPEECH2015}

\usepackage[utf8]{inputenc}
\DeclareUnicodeCharacter{0181}{\Ɓ}
\DeclareUnicodeCharacter{0253}{\ɓ}
\DeclareUnicodeCharacter{018A}{\Ɗ}
\DeclareUnicodeCharacter{0257}{\ɗ}
\DeclareUnicodeCharacter{0198}{\Ƙ}
\DeclareUnicodeCharacter{0199}{\ƙ} 

\usepackage{graphicx}
\usepackage{amssymb,amsmath,bm}
\usepackage{textcomp}

\def\vec#1{\ensuremath{\bm{{#1}}}}
\def\mat#1{\vec{#1}}


\sloppy % better line breaks
\ninept

\title{Collaborative Annotation for Person Identification in TV Shows}

\makeatletter
\def\name#1{\gdef\@name{#1\\}}
\makeatother \name{\em Matheuz Budnik$^1$, Laurent Besacier$^1$, Johann Poignant$^2$, Hervé Bredin$^2$, Claude Barras$^2$,  \\
Mickael Stefas$^3$, Pierrick Bruneau$^3$, Thomas Tamisier$^3$}

\address{$^1$Laboratoire d'Informatique de Grenoble (LIG), Univ. Grenoble Alpes, Grenoble, France \\
  $^2$LIMSI, CNRS - Orsay, France \\
  $^3$LIST, Luxembourg \\
  {\small \tt adresses mail} 
}

\begin{document}
  \maketitle
  %
  \begin{abstract}
This paper presents 
  \end{abstract}
  \noindent{\bf Index Terms}: to be added

  \section{Introduction}
      \subsection{Demo presented}
(Décrire en qq lignes la démo) 

Resp : all

 \subsection{Camomile project}
reprendre des bouts de texte déjà écrits pour présenter le projet

Resp : all

      \section{Collaborative annotation using Camomile tools}

mettre / présenter une vue d'ensemble (schéma) 

      \subsection{Collaborative annotation framework}
(présenter le framework et pointer sur le lien github précis - passer rapidement - reprendre anciens articles)

Resp : Johann
  
      \subsection{Web annotation front-end}
(présenter le front-end web et pointer sur le lien github précis - y passer plus de temps 

Resp : LIST

      \subsection{Active learning backend}
(présenter le l'approche LIG - retraining and adaptation - et pointer sur le lien github précis - y passer plus de temps 

Resp : Matheusz

      \subsection{Compatibility with other annotation tools}
voir si on ajoute une partie comme ça ou pas?

Resp : LIMSI



  \section{Dry run evaluation}
      \subsection{Use case : multimodal speaker annotation}
     
décrire la tâche et les participants à l'expe
      
      \subsection{Quantitative analysis}
    
stats présentés par Matheusz au meeting de Madrid

Resp : Mateusz
  
     \subsection{Qualitative analysis}
    
Analyse des resultats du Survey

resp : LIST




  \section{Conclusion}
  
    \subsection{Supporting data files}

description vidéo démo soumise en même temps que le papier
résumé code github pointé ?    

resp : All

    \subsection{Live demo scenario}

décrire précsément ce qui sera présenté à Dresde en Sept 2015

  
  \newpage
  \eightpt
  \bibliographystyle{IEEEtran}
  
  \bibliography{Camomile}

\end{document}
